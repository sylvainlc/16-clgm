\documentclass[nolayout]{article}

\usepackage[utf8]{inputenc}
\usepackage[english]{babel}
\usepackage{amsmath,amsthm}
\usepackage{geometry}
\usepackage[textwidth=4cm,textsize=footnotesize]{todonotes}
\usepackage{fancyhdr}
\pagestyle{fancy}
\usepackage{amssymb,color,bbm,xargs}
\usepackage{graphicx}
\usepackage[active]{srcltx}
\usepackage{ifthen}
\usepackage{enumerate}
\usepackage{color}
\usepackage{accents}
\usepackage{dsfont}
\usepackage[ruled,vlined]{algorithm2e}
\usepackage{subfig}
\usepackage{multirow}
\usepackage{placeins}

\usepackage{aliascnt}
\theoremstyle{plain}
\newtheorem{theorem}{Theorem}
\newtheorem{assumption}{H\hspace{-3pt}}
\newtheorem*{assumption*}{A}
\newtheorem{assumptionDelta}{H\hspace{-3pt}(\Delta)}
\newtheorem{assumptionsmallSetD}{H\hspace{-3pt}(\smallSetD \times \smallSetD)}
\newtheorem{assumptionAR}{AR\hspace{-3pt}}
\newtheorem{assumptionCN}{CN\hspace{-3pt}}


\newaliascnt{proposition}{theorem}
\newtheorem{proposition}[proposition]{Proposition}
\aliascntresetthe{proposition}

\newaliascnt{lemma}{theorem}
\newtheorem{lemma}[lemma]{Lemma}
\aliascntresetthe{lemma}
\newaliascnt{corollary}{theorem}
\newtheorem{corollary}[corollary]{Corollary}
\aliascntresetthe{corollary}

\newtheorem{hypothese}{Hypothesis}
\newtheorem{fact}[theorem]{Fact}


\theoremstyle{definition}
\newaliascnt{definition}{theorem}
\newtheorem{definition}[definition]{Definition}
\aliascntresetthe{definition}

\newaliascnt{remark}{theorem}
\newtheorem{remark}[remark] {Remark}
\aliascntresetthe{remark}



\usepackage{hyperref}
\usepackage[nameinlink,noabbrev]{cleveref}

\providecommand*{\definitionautorefname}{Definition}
\providecommand*{\lemmaautorefname}{Lemma}
\providecommand*{\exampleautorefname}{Example}
\providecommand*{\propositionautorefname}{Proposition}
\providecommand*{\procautorefname}{Procedure}
\providecommand*{\exerciseautorefname}{Exercise}
\providecommand*{\corollaryautorefname}{Corollary}
\providecommand*{\remarkautorefname}{Remark}
\providecommand*{\algorithmautorefname}{Algorithm}
\providecommand*{\assumptionautorefname}{H\hspace*{-3pt}}  
\providecommand*{\assumptionARautorefname}{AR\hspace*{-3pt}}
\providecommand*{\assumptionCNautorefname}{CN\hspace*{-3pt}}


\newcommand*{\largedot}{\cdot}

\def\Rset{\mathbb{R}}
\def\Nset{\mathbb{N}}
\def\nset{\mathbb{N}}
\def\rset{\mathbb{R}}
\def\varb{b}
\newcommand{\1}{\mathbbm{1}}

\newcommand{\floor}[1]{\left\lfloor #1 \right\rfloor}
\newcommand{\ceil}[1]{\left\lceil #1 \right\rceil}

\def\ie{\textit{i.e.}}
\newcommand{\coint}[1]{\left[#1\right)}
\newcommand{\ocint}[1]{\left(#1\right]}
\newcommand{\ooint}[1]{\left(#1\right)}
\newcommand{\ccint}[1]{\left[#1\right]}
\newcommand{\Wienerspace}{\mathbf{W}}

\def\PE{\mathbb{E}}
\newcommandx{\PEt}[2][1=]{\mathbb{E}_{#1}\left[#2\right]}
\def\rmd{\mathrm{d}}
\def\rme{\mathrm{e}}
\def\target{\pi}
\def\dim{d}
\def\score{\accentset{\largedot}{\ell}}
\def\Vdot{\dot{V}}
\def\unitarget{f}
\def\rmL{\mathrm{L}}
\def\dimy{\mathsf{p}}
\def\dimz{\mathsf{m}}
\def\bfX{\boldsymbol{X}}
\def\xbf{\boldsymbol{x}}
\def\ybf{\boldsymbol{y}}
\def\Xbf{\boldsymbol{X}}
\def\G{\mathcal{G}}
\def\barG{\overline{G}}
\def\barH{\overline{H}}
\def\Gammabf{\mathbf{\Gamma}}
\def\Ybf{\boldsymbol{Y}}
\newcommand{\norm}[1]{\left\Vert #1 \right \Vert}
\newcommand{\plusinfty}{+ \infty}
\newcommand{\defEns}[1]{\left \{ #1 \right \}}
\newcommand{\abs}[1]{\left\vert #1 \right\vert}
\def\Zbf{\boldsymbol{Z}}
\def\eqsp{\,}
\def\leb{\lambda}
\def\limd{\underset{d\to +\infty}{\longrightarrow}}
\newcommand{\normMat}[2]{\left\|#2\right\|_{#1}}



\newcommand{\parenthese}[1]{\left( #1 \right)}
\newcounter{hypH}
\newenvironment{hypH}{\refstepcounter{hypH}\begin{itemize}
\item[{\bf H\arabic{hypH}}]}{\end{itemize}}
\newcommand{\eqdef}{\ensuremath{:=}}
\newcommand{\setAccept}{\mathcal{A}}
\def\iid{\operatorname{i.i.d.}}
\newcommand{\sachant}[1]{\left| #1 \right.}
\newcommand{\setDisconDotV}{\mathcal{D}_{\dot{V}}}

\def\restriction#1#2{\mathchoice
              {\setbox1\hbox{${\displaystyle #1}_{\scriptstyle #2}$}
              \restrictionaux{#1}{#2}}
              {\setbox1\hbox{${\textstyle #1}_{\scriptstyle #2}$}
              \restrictionaux{#1}{#2}}
              {\setbox1\hbox{${\scriptstyle #1}_{\scriptscriptstyle #2}$}
              \restrictionaux{#1}{#2}}
              {\setbox1\hbox{${\scriptscriptstyle #1}_{\scriptscriptstyle #2}$}
              \restrictionaux{#1}{#2}}}
\def\restrictionaux#1#2{{#1\,\smash{\vrule height .8\ht1 depth .85\dp1}}_{\,#2}}

\newcommandx\sequence[3][2=,3=]
{\ifthenelse{\equal{#3}{}}{\ensuremath{\{ #1_{#2}\}}}{\ensuremath{( #1_{#2}, \eqsp #2 \in #3 )}}}
\newcommandx{\sequencen}[2][2=n\in\nset]{\ensuremath{( #1, \eqsp #2 )}}
\newcommandx\dsequence[4][3=k,4=\zset]{\ensuremath{( (#1_{#3}, #2_{#3}), \eqsp #3 \in #4 )}}
\newcommandx{\as}[1][1=\PP]{\ensuremath{#1\--\mathrm{a.s.}}}
\newcommand{\eqspp}{\ \ }

\def\canonicalFiltration{\mathscr{B}}

\def\tildeX{\widetilde{X}}
\def\tildeY{\widetilde{Y}}
\def\barX{\bar{X}}
\def\barY{\bar{Y}}
\def\barS{\bar{S}}
\newcommand{\coupling}[2]{\mathcal{C}(#1,#2)}
\newcommand\proba[1]{\mathbb{P}\left[ #1 \right]}
\newcommand\expe[1]{\mathbb{E}\left[ #1 \right]}
\def\osc{\operatorname{osc}}
\def\borel{\mathcal{B}}
\def\mcf{\mathcal{F}}